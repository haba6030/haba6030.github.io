\documentclass[11pt]{article}

\usepackage[T1]{fontenc}
\usepackage[utf8]{inputenc}
\usepackage{lmodern}

\usepackage[a4paper,margin=1.8cm]{geometry}
\setlength{\parindent}{0pt}
\setlength{\parskip}{0.25em}

\usepackage{tabularx}
\usepackage{array}
\usepackage{enumitem}
\usepackage{hyperref}
\hypersetup{colorlinks=true,urlcolor=black}
\setlist[itemize]{nosep,leftmargin=1.2em}

\newcolumntype{L}{>{\raggedright\arraybackslash}X}
\newcolumntype{R}{>{\raggedleft\arraybackslash}p{0.28\textwidth}}

\newcommand{\orgrow}[2]{%
  \begin{tabularx}{\textwidth}{@{}L R@{}}%
  \textbf{#1} & #2\\
  \end{tabularx}
}
\newcommand{\rolerow}[2]{%
  \begin{tabularx}{\textwidth}{@{}L R@{}}%
  #1 & #2\\
  \end{tabularx}\vspace{0.15em}
}
\newcommand{\cvsection}[1]{%
  \vspace{0.9em}
  {\large\bfseries\MakeUppercase{#1}}\par
  \vspace{0.15em}\hrule height 0.6pt \vspace{0.4em}
}
\newcommand{\cvname}[1]{\vspace{-0.2em}{\LARGE\bfseries #1}\par\vspace{0.3em}}
\newcommand{\cvline}[1]{#1\par}

\begin{document}

% ===== Header =====
\cvname{Jinil Kim}
\cvline{\href{mailto:haba6030@snu.ac.kr}{haba6030@snu.ac.kr} \quad | \quad +82-10-5264-9444 \quad| \quad Seoul, Korea } 
\cvline{LinkedIn: \url{linkedin.com/in/jinilkimpotato/} \quad | \quad GitHub: \url{github.com/haba6030}}

% ===== Research Interest =====
\cvsection{Research Interest}
Clarifying causal factors of individual difference via Neuro-Decoding and Computational Modeling

% ===== Education =====
\cvsection{Education}
\orgrow{Seoul National University}{Seoul, South Korea}
\rolerow{B.A. in Linguistics, B.A. in Psychology, B.S. in Computer Science}{Mar 2021 -- Exp. Feb 2027}
GPA: 4.17 / 4.3 \quad | \quad Major (4.0): Ling./Psych.: 4.00, CS: 3.82\\

\orgrow{The University of Hong Kong}{Hong Kong SAR}
\rolerow{Summer Research Program, Faculty of Psychology}{Jun 2025 -- Aug 2025}

% ===== Research Experience =====
\cvsection{Research Experience}
\orgrow{Mental health, Internet, Neuroscience \& Decision-making Lab}{University of Hong Kong}
\rolerow{Undergraduate Researcher \ \ (Advisor: Prof.\ Yuan-Wei Yao)}{Jun 2025 -- Aug 2025}
\begin{itemize}
  \item Developed and compared hierarchical Bayesian RL--DDM variants to model emotion and reward prediction errors in social decision-making.
  \item Introduced a time-varying Drift Diffusion Model (tDDM) to study how emotion-based and reward-based prediction errors dynamically shape social decision-making
\end{itemize}

\orgrow{Computational Clinical Science Laboratory}{Seoul National University}
\rolerow{Undergraduate RA \ \ (Advisor: Dr.\ Woo-Young Ahn)}{Apr -- Jun 2025; Sep 2025 -- }
\begin{itemize}
  \item Implemented RL-based behavioral models to dissociate stable trait parameters from dynamic state-dependent fluctuations.
  \item Contributed to a multimodal project integrating EEG, behavioral, and app-based digital phenotype data for mental health risk modeling.
\end{itemize}

% ===== Ongoing Projects =====
\cvsection{Ongoing Projects}
\textbf{Development of a Personalized Color Vision Correction Display Filter Using fMRI-Based Neural Responses and Deep Learning}\\
\rolerow{Project Lead}{Apr 2025 -- Present}
\begin{itemize}
  \item Applied ROI-based decoding, forward encoding models, and Procrustes alignment to identify individual-specific neural color spaces.
  \item Designed the full research pipeline: experiment design, data preprocessing, neural modeling, and optimization-based filter generation.
  \item Funded by SNU Undergraduate Research Grant (KRW 5,000,000).
\end{itemize}

\textbf{Time-Varying Drift Diffusion Modeling of Emotion and Reward in Depression}\\
\rolerow{Project Lead}{Jun 2025 -- Present}
\begin{itemize}
  \item Introduced a time-varying Drift Diffusion Model (tDDM) to study how emotion-based and reward-based prediction errors dynamically shape social decision-making
\end{itemize}

% ===== Poster Presentations =====
\cvsection{Poster Presentations}
\begin{itemize}
  \item Kim, Jinil, Yao, Y. W. (Aug 2025). \textit{Why Do Depressed Individuals Punish?: A Computational Investigation of Emotion Learning and Bias in Maladaptive Social Behavior.} Poster presented at the Summer Research Program Poster Session, University of Hong Kong.
\end{itemize}

% ===== Honors & Awards =====
\cvsection{Honors \& Awards}
\orgrow{Humanity 100 Years Undergraduate Scholarship}{Korea Student Aid Foundation}
\rolerow{Amount: KRW 17{,}000{,}000 (USD 12{,}500)}{Mar 2025 -- Jul 2026}


\orgrow{The University of Hong Kong Foundation for Educational Development and Research Scholarship}{The University of Hong Kong}
\rolerow{Amount: HKD 10{,}000 (USD 1{,}284)}{Jun 2025 -- Aug 2025}

\orgrow{Grand prize in SNU Liberal Arts Competition}{Seoul National University}
\rolerow{Title: \textit{Interdisciplinary exploration with chatbots}}{Jul 2023}

\orgrow{Megastudy Student Achievement Scholarship}{MegaStudy}
\rolerow{Amount: KRW 3{,}000{,}000 (USD 2{,}000)}{Mar 2021 -- Mar 2022}

\orgrow{Superior Academic Performance scholarship}{Seoul National University}
\rolerow{}{Mar 2022, 2023; Sep 2022}

% ===== Competitions =====
\cvsection{Competitions}

\orgrow{Ambient AI Competition}{Seoul National University}
\rolerow{Project: \textit{Fashion Recommendation AI}}{Dec 2022 -- Jan 2022}
\begin{itemize}
  \item Vectorized fashion dataset and extracted feature from image input by using VGG16 model. 
  \item Finetuned MLP Model for recommendation
\end{itemize}

% ===== Skills =====
\cvsection{Skills}
\textbf{Computational Modeling:} Bayesian hierarchical models, RL, DDM, RLDDM, mixed-effects models\\
\textbf{Neuroimaging:} fMRI preprocessing, ROI-based decoding, forward encoding models, EEG connectivity analysis\\
\textbf{Programming:} Python, R, Stan, C++, Java\\
\textbf{ML \& Data:} Deep learning (MLP, CNN), high-dimensional time-series, multimodal data integration\\

% ===== Languages =====
\cvsection{Languages}
Korean (Native), \ \ English (Full Professional Proficiency), \ \ French (Beginner)

\end{document}
